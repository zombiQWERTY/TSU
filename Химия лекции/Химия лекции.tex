\documentclass[a4paper, 12pt, oneside]{article}
\usepackage[left=2cm,right=2cm,top=2cm,bottom=2cm,bindingoffset=0cm]{geometry}
\geometry{a4paper}
\usepackage{ucs,graphicx,amssymb,amsmath,mathtext}

\usepackage{tabularx}
\usepackage[utf8x]{inputenc}
\usepackage[russian]{babel}

\sloppy


\title{Химия лекции}
\author{Павел Зиновьев}

\begin{document}
\maketitle
\begin{abstract}
	Математик – это тот, кто умеет находить аналогии между утверждениями. Лучший 
	математик – кто устанавливает аналогии доказательств. Более сильный может заметить 
	аналогии теорий. Но есть и такие, кто между аналогиями видит аналогии.
	\begin{flushright}
		 Стефан Банах \copyright
	\end{flushright}
\end{abstract}
\newpage
\tableofcontents
\contentsname
\newpage


\section{Информация}
Преподаватель: Комарин Станислав Сергеевич).\\
\subsection{Основы химической термодинамики}
Химическая термодинамика изучает энергетические явления химических процессов и устанавливает направление и пределы протекания процессов при заданных условиях\\
Система - это вещество или смесь веществ мысленно или фактически выделенное из окружающей среды. Может быть как изолированный раствор в стакане\\
Типы систем:
\begin{itemize}
	\item Открытые системы обмениваются с окружающей средой веществом и энергией
	\item Закрытые системы способны обменивать с окружающей средой только энергией, но не веществом (чай Nestea)
	\item Изолированные системы не обмениваются с окружающей средой ни энергией, ни веществом (термос с чаем)
\end{itemize}
Системы по фазовому состоянию:
\begin{itemize}
	\item Гомогенные системы состоят из одной фазы (твердые сплавы, воздух)
	\item Гетерогенные системы состоят из двух и более фаз
\end{itemize}
{\bfseries Фаза} --- это гомогенная часть системы (однородная) с одинаковыми во всех точках свойствами и отделенная от других частей системы поверхностями раздела\\
Системы могут находится в различных состояниях, которые определяются параметрами системы\\
{\bfseries Параметр} --- любое свойство системы (масса, температура)\\
{\bfseries Функция состояния системы} --- энергетические характеристики системы, которые зависят только от параметров состояния системы и не зависит от того, каким способом это состояние было достигнуто\\
{\bfseries Внутренняя энергия системы} ($U$) --- кинетическая энергия движения и потенциальная энергия взаимодействия всех частиц, образующих систему\\
{\bfseries Внутренняя энергия} --- функция состояния системы. Абсолютное значение внутренней энергии не поддается определению\\
$\varDelta U$ --- изменение внутренней энергии\\
{\bfseries Теплота} --- микроскопическая форма передачи энергии путем хаотического движения частиц. Теплота не имеет направления передачи энергии (передается по всем направлениям)\\
{\bfseries Работа} --- макроскопическая форма передачи энергии. Происходит путем перемещения масс под действием сил. Работа имеет направление.\\
Измеряется в джоулях (Дж)\\
{\bfseries Теплота} и {\bfseries работа} --- функции пути\\
\newpage
{\bfseries Первый закон термодинамики} --- поглощенная системой теплота расходуется на увеличение внутренней энергии системы и совершение системой работы:
\begin{gather*}
	Q=\varDelta U+A\\
	A = A'+p\varDelta N\\
	A' = 0\\
	Q = \varDelta U+p\varDelta V = \varDelta U+p(V_0-V_1)=\varDelta U
\end{gather*}
$P = const$ --- изобарный процесс\\
$V = const$ --- изохорный процесс\\
$T = const$ --- изотермический процесс
\begin{gather*}
	Q = (U_1-U_0)+p(V_1-V_0)\\
	Q=(U_1+pV_1)-(U_0+pV_0) = H_1-H=\varDelta H\\
\end{gather*}
$U+pV=H$ --- энтальпия реакции
$$
	\varDelta H = \varDelta U +p\varDelta V (Дж)
$$
{\bfseries Закон Гесса} --- тепловой эффект химической реакции, протекающий при постоянном объеме и температуре, не зависит от пути процесса, определяется только начальным и конечным состоянием реагирующих веществ (какие были реагенты --- какие получились продукты)
\begin{gather*}
	A_1, A_2, \dots \longrightarrow^{\varDelta H} B_1, B_2 \dots\\
	\downarrow^{\varDelta H_1}\\
	C_1, C_2, \dots\\
	\downarrow^{\varDelta H}\\
	D_1, D_2, \dots\\
	goto0\\\\
	\varDelta H = \varDelta H_1 + \varDelta H_2 \varDelta H_3
\end{gather*}
Тепловой эффект процесса равен сумме эффектов его стадий\\\\
{\bfseries Второе следствие закона Гесса} --- тепловой эффект химической реакции равен разности суммы теплот образования продуктов реакции и исходных веществ реакции с учетом их стехиометрических коэффициентов.
$$
\varDelta H = (a\times \varDelta H(c) + d\times \varDelta H(0))-(a \varDelta H(A)+b\times \varDelta H(B))
$$
\newpage
{\bfseriesЗакон Гесса:}
\begin{gather*}
	p=101.325 (кПа)\\
	T=238 (K)\\
	c = 1 (моль/л)
\end{gather*}
Энтальпия образования простых веществ равна нулю
$\varDelta H>0$ --- эндотермический\\
$\varDelta H<0$ --- экзотермический\\
$\varDelta H = H_2-H_1$\\
$N_2+3H_2 \longrightarrow 2NH_2$\\

\subsection{Самопроизвольный и несамопроизвольный процессы}

Самопроизвольные процессы --- это те процессы, протекание которых происходит без затрат энергии\\
Несамопроизвольный процессы --- это те процесс, для протекания которых нужно затратить энергию, либо совершить работу\\\\
\fbox{Ar}\fbox{Ne} $\rightarrow$ \fbox{Ar $\longleftrightarrow$ Ne}\\

Термодинамическая вероятность - число микросостояний, с помощью которых может быть реализовано данное макрососотояние системы\\
$S=Rl_nW$\\
$R=8.314$ (Дж/(моль $\times$ К))\\
{\bfseries Энтропия} --- мера хаоса\\
Свойства энтропии:\\
Изменение энтропии процесса равно сумме энтропий каждой из ее стадий. Энтропия подчиняется закону Гесса и его следствиям\\
Энтропия идеального кристалла при 0К равняется  0\\
Управление протеканием процесса определяется двумя противоположными тенденциями. Система стремится перейти в состояние с наименьшей энергией и в состояние с наибольшей энтропией\\
Энергия Гиббса ($G$)
$$G=H-TS$$
$$\varDelta G = \varDelta H-T\varDelta S$$
$$\varDelta G = -A_{max}$$
$\varDelta G<0$ - самопризвольный\\
$\varDelta G>0$ - несамопризвольный\\
$\varDelta G=0$ - равновесие\\



































































\end{document}
