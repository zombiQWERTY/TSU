\documentclass[a4paper, 11pt, oneside]{article}
\usepackage[left=2cm,right=2cm,top=2cm,bottom=2cm,bindingoffset=0cm]{geometry}
\geometry{a4paper}
\usepackage{ucs,graphicx,amssymb,amsmath,mathtext}

\usepackage{tabularx}
\usepackage[utf8x]{inputenc}
\usepackage[russian]{babel}

\sloppy


\title{Введение в физику}
\author{Зиновьев Павел}

\begin{document}
\maketitle
\begin{abstract}
	Если тебя квантовая физика не испугала, значит, ты ничего в ней не понял. 
	\begin{flushright}
		Нильс Бор \copyright 
	\end{flushright}
\end{abstract}
\newpage
\tableofcontents
\contentsname
\newpage

\section{Информация}
Якунова Елена Викторовна (каб. 9-518 или 9-514)\\
С собой иметь таблицу интегралов, производных, правила работы со степенями

\subsection{Материалы}
На кафедре физики ТулГУ $\rightarrow$ введение в физику
\subsection{Зачеты}
В конце семестра зачет без оценки\\
Зачетный балл 40. Можно набрать в течении семестра посещаемостью, 3 контрольные работы (2 обязательные), домашняя работа (по вызову будут проверять). Аттестация неофициальная в конце года\\
Входит в курс:\\
Математическая составляющая физики, интегралы, дифференциалы, векторная алгебра. 

\subsection{Структура курса физики}


\begin{tabularx}{\textwidth}{|p{2.96cm}|*{5}{p{2.96cm}|}}
	\hline
	\multicolumn{5}{|c|}{\bfseries{Структура курса физики}} \\
	\hline
	Механика& Молекулярная физика и термодинамика& Электричество и магнетизм& Колебания и волны& Квантовая физика \\
	\hline
	Кинематика& МКТ& Электростатика& Механические колебания и волны& Квантовая оптика \\
	\hline
	Динамика& Статистическая физика& Электродина-мика& Электрические колебания и волны& Квантовая механика \\
	\hline
	Статика& Термодинамика& Электромагнит-ное поле& Волновая оптика& Строение атома \\
	\hline
	Классическая механика& & Постоянный и переменный электрический ток& & Физика твердого тела \\
	\hline
	Релятивистская механика& & Магнитостатика& & Физика атомного ядра \\
	\hline
	Механика сплошных сред& & Электромагнит-ная индукция& & \\
	\hline
	Гидродинамика& & & & \\
	\hline
	Акустика& & & & \\
	\hline
\end{tabularx}
Физические основы механики\\
Два вида неживой материи - вещество и поле\\
Движение - это всевозможные изменения материи\\
Форма бытия материи - это пространство и время\\
Механическое движение - изменение положения в пространстве с течением времени\\
Статика изучает законы равновесия системы тел
\newpage

\section{Кинематика}
Основной задачей кинематики является нахождение положения тела в любой момент времени, если известна
 $x_0, y_0, V_0, a_0$
\section{Задание на дом}
Вспомнить что такое вектора и производные. 























\end{document}
