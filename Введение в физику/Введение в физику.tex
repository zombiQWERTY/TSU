\documentclass[a4paper, 12pt, oneside]{article}
\usepackage[left=2cm,right=2cm,top=2cm,bottom=2cm,bindingoffset=0cm]{geometry}
\geometry{a4paper}
\usepackage{ucs,graphicx,amssymb,amsmath,mathtext}

\usepackage{tabularx}
\usepackage[utf8x]{inputenc}
\usepackage[russian]{babel}

\sloppy


\title{Введение в физику}
\author{Зиновьев Павел}

\begin{document}
\maketitle
\begin{abstract}
	Если тебя квантовая физика не испугала, значит, ты ничего в ней не понял. 
	\begin{flushright}
		Нильс Бор \copyright 
	\end{flushright}
\end{abstract}
\newpage
\tableofcontents
\contentsname
\newpage

\section{Информация}
Якунова Елена Викторовна (каб. 9-518 или 9-514)\\
С собой иметь таблицу интегралов, производных, правила работы со степенями

\subsection{Материалы}
На кафедре физики ТулГУ $\rightarrow$ введение в физику
\subsection{Зачеты}
В конце семестра зачет без оценки\\
Зачетный балл 40. Можно набрать в течении семестра посещаемостью, 3 контрольные работы (2 обязательные), домашняя работа (по вызову будут проверять). Аттестация неофициальная в конце года\\
Входит в курс:\\
Математическая составляющая физики, интегралы, дифференциалы, векторная алгебра. 

\subsection{Структура курса физики}


\begin{tabularx}{\textwidth}{|p{2.96cm}|*{5}{p{2.96cm}|}}
	\hline
	\multicolumn{5}{|c|}{\bfseries{Структура курса физики}} \\
	\hline
	Механика& Молекулярная физика и термодинамика& Электричество и магнетизм& Колебания и волны& Квантовая физика \\
	\hline
	Кинематика& МКТ& Электростатика& Механические колебания и волны& Квантовая оптика \\
	\hline
	Динамика& Статистическая физика& Электродина-мика& Электрические колебания и волны& Квантовая механика \\
	\hline
	Статика& Термодинамика& Электромагнит-ное поле& Волновая оптика& Строение атома \\
	\hline
	Классическая механика& & Постоянный и переменный электрический ток& & Физика твердого тела \\
	\hline
	Релятивистская механика& & Магнитостатика& & Физика атомного ядра \\
	\hline
	Механика сплошных сред& & Электромагнит-ная индукция& & \\
	\hline
	Гидродинамика& & & & \\
	\hline
	Акустика& & & & \\
	\hline
\end{tabularx}
Физические основы механики\\
Два вида неживой материи - вещество и поле\\
Движение - это всевозможные изменения материи\\
Форма бытия материи - это пространство и время\\
Механическое движение - изменение положения в пространстве с течением времени\\
Статика изучает законы равновесия системы тел
\newpage

\section{Кинематика}
Основной задачей кинематики является нахождение положения тела в любой момент времени, если известна
 $x_0, y_0, V_0, a_0$

\subsection{Физические модели}
\subsubsection{Радиус вектора}
Материальная точка: масса есть, размеров нет\\
Абсолютно твердое тело: две материальные точки в теле, расстояние между которыми не изменяется ни при каких обстоятельствах\\
\subsubsection{Векторный способ описания движений частиц}
Радиус вектор --- это вектор, который начинается в точке начала отсчета координат и заканчивается на точке. При перемещении по любой траектории радиус-вектор меняет свои координаты и длину. Таким образом радиус-вектор --- функция времени: $\overrightarrow{r}(t)$\\
Единичные орты (вектора) --- вектор, направленный вдоль координатной оси. Длина вектора равна единице\\
$\overrightarrow{r}(t)=\overrightarrow{i} r_x+\overrightarrow{j} r_y+\overrightarrow{k} r_z$\\
$\overrightarrow{r}(t)=\overrightarrow{i} x(t)+\overrightarrow{j} y(t)+\overrightarrow{k} z(t)$\\

Перемещение за единицу времени ---
$$
	\overrightarrow{V}(t)=\frac{d\overrightarrow{r}}{dt}
$$
Изменение скорости за единицу времени ---
$$
	\overrightarrow{a}(t)=\frac{d^2r}{dt^2}
$$
$$
	\overrightarrow{V}(t)=\overrightarrow{i}V_x(t)+\overrightarrow{j}V_y(t)+\overrightarrow{k}V_z(t)
$$
$$
	\overrightarrow{a}(t)=\overrightarrow{i}a_x(t)+\overrightarrow{j}a_y(t)+\overrightarrow{k}a_z(t)
$$
Если известны: $x(t),y(t),z(t)$, то можно определить проекции скорости на оси: $V_x(t)=\frac{dx}{dt}, V_y(t)=\frac{dy}{dt}, V_z(t)=\frac{dz}{dt}$\\\\
Ускорение: $a_x(t)=\frac{dV_x}{dt}, a_y(t)=\frac{dV_y}{dt}, a_z(t)=\frac{dV_z}{dt}$\\
\begin{gather*}
	r=\sqrt{x^2+y^2+z^2}\\
	V=\sqrt{V^2+V^2+V^2}\\
	a=\sqrt{a^2+a^2+a^2}
\end{gather*}
\newpage
\subsection{Прямая задача кинематики}
Радиус-вектор частицы зависит от времени по закону:
$$
\overrightarrow{r}(t)=\overrightarrow{i}A\frac{t}{\tau}+\overrightarrow{j}B(\frac{t}{\tau})^2
$$
Найти тангенс угла между вектором скорости и осью $x$ в момент времени $t=\tau=1с$, $A=B=C=1м$\\
$$
	\tg\alpha = \frac{V_y}{V_x}
$$
$$
	V_y=\frac{B(\frac{t}{\tau})^2}{dt} = \frac{B}{\tau^2}2t=2
$$
$$
\overrightarrow{r}(t)=\overrightarrow{i}A\frac{t}{\tau}+\overrightarrow{j}B(\frac{t}{\tau})^2+\overrightarrow{k}C(\frac{t}{\tau})^3
$$
$$
V=\sqrt{V_x^2+V_y^2+V_z^2}=3.74
$$

\subsection{Обратная задача кинематики}

Если известны зависимости $a_x(t), a)y(t),a_z(t)$, $x_0, y_0, z_0, V_{0x},V_{0y}, V_{oz}$\\
$V_x=V_{0x}+\int_0^t a_y(t)dt$\\
$V_y=V_{0y}+\int_0^t a_y(t)dt$\\
$V_z=V_{0a}+\int_0^t a_y(t)dt$\\
$z=z_0+\int_0^t V_z(t)dt$
Сайт Семин §Прямая задача по механике(тема 1) решить 12 задач













































\end{document}
