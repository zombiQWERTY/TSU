\documentclass[a4paper, 11pt, oneside]{article}
\usepackage[left=2cm,right=2cm,top=2cm,bottom=2cm,bindingoffset=0cm]{geometry}
\geometry{a4paper}
\usepackage{ucs,graphicx,amssymb,amsmath,mathtext}

\usepackage{tabularx}
\usepackage[utf8x]{inputenc}
\usepackage[russian]{babel}

\sloppy


\title{Математика лекции}
\author{Павел Зиновьев}

\begin{document}
\maketitle
\begin{abstract}
	Математик – это тот, кто умеет находить аналогии между утверждениями. Лучший 
	математик – кто устанавливает аналогии доказательств. Более сильный может заметить 
	аналогии теорий. Но есть и такие, кто между аналогиями видит аналогии.
	\begin{flushright}
		 Стефан Банах \copyright
	\end{flushright}
\end{abstract}
\newpage
\tableofcontents
\contentsname
\newpage


\section{Информация}
Преподаватель: Соколова Марина Юрьевна (89105515906). Кафедра математического моделирования (каб. 12-314)\\
modeling.tula.ru --- сайт кафедры
\subsection{Книги}
Бугров Я. С. и Никольский С. Н. --- высшая математика в 3-х томах (на 1 курс нужны 1 и 2 тома)\\
Пискунов Н. С. --- дифференциальное и интегральное исчисления в 2-х томах\\
Аверин В. В., Соколова М. Ю., Христич В. М. --- Математика. Курс лекций (2010)\\
Кузнецов Л. С. --- Сборник заданий по высшей математике\\
Данко П. Е., Попов А. П., Кожевникова Т. Я. --- Высшая математика в упражнениях и задачах в 2-х томах (на 1 курс нужен 1 том)

\section{Линейная алгебра и аналитическая геометрия}
\subsection{Матрицы и определители}
План:
\begin{enumerate}
	\item Определение матрицы
	\item Определение и свойства определителей
	\item Ранг матрицы
	\item Действия над матрицами
	\item Некоторые специальные виды матриц
\end{enumerate}
\subsubsection{Определение матрицы}
Матрицей называется совокупность чисел, расположенных в прямоугольной таблице, состоящей из $m$ строк и $n$ столбцов.\\
Обозначается так: $(A)$ или $[A]$\\

\begin{gather*}
	[A] =
	\begin{pmatrix}
		a_{11}& a_{12}& a_{13}& a_{1n}\\
		a_{21}& a_{22}& a_{23}& a_{2n}\\
		a_{31}& a_{32}& a_{33}& a_{3n}\\
		\dots & \dots & \dots & \dots\\
		a_{mn}& a_{mn}& a_{mn}& a_{mn}
	\end{pmatrix}
\end{gather*}
Это матрица размера $m \times n$ и является прямоугольной. Если $m = n$, то такая матрица называется квадратной матрицей порядка $m$.\\
Если матрица имеет только один столбец, то она называется матрицей столбцов (матрица-столбец) $n = 1$, $m$ --- высота столбца
\begin{gather*}
	[A] =
	\begin{pmatrix}
		a_{11}\\
		a_{21}\\
		a_{31}\\
		\dots\\
		a_{m2}
	\end{pmatrix}
	=
	\begin{pmatrix}
		a_{1}\\
		a_{2}\\
		a_{3}\\
		\dots\\
		a_{m}
	\end{pmatrix}
\end{gather*}

Если матрица имеет только один столбец, то она называется матрицей строк (матрица-строка) $m = 1$, $n$ --- длина строки\\
Два столбца называют равными, если они имеют одинаковую высоту и равные элементы с соответствующими номерами\\
Две строки называют равными, если они имеют равную длину и равные элементы с соответствующими номерами\\
Две прямоугольные матрицы равны, если они имеют одинаковые размеры и равные строки и столбцы с одинаковыми номерами\\
\subsubsection{Определение и свойства определителей}
Для любой квадратной матрицы порядка $m$ вводят понятие определителя (детерминанта)\\
Обозначается: $\det A$ или $|A|$ или
\begin{gather*}
	\begin{vmatrix}
		a_{11}& a_{12}& a_{1n}\\
		a_{21}& a_{22}& a_{2n}\\
		a_{31}& a_{32}& a_{3n}\\
		\dots & \dots & \dots\\
		a_{mn}& a_{mn}& a_{mn}
	\end{vmatrix}
\end{gather*}
Определителем матрицы $m$ порядка $n>1$ называют число, равное
\begin{gather*}
	|A| = \sum_{k=1}^n (-1)^{1+k} a_{ak}M^1_k
\end{gather*}
$M_1^k$ называют дополнительным минором элемента $a_k^1$\\
%Что то пропустил
При $n=1$ получаем определитель 1-го порядка. 
\begin{gather*}
	n=2\\
	\begin{vmatrix}
		a_{11}& a_{12}\\
		a_{21}& a_{22}\\
		a_{31}& a_{32}
	\end{vmatrix}
	=
	\sum_{k=1}^2 (-1)^{1+2} a_{1k}M^1_k = (-1)^{1+1} a_{11} |a_{22}| + (-1)^{1+2}a_{12} |a_{21}|=a_{11}a_{12}-a_{12}a_{21}
\end{gather*}\\
При $n=3$ получаем определитель 3-го порядка. 
\begin{gather*}
	n=3\\
	\begin{vmatrix}
		a_{11}& a_{12}& a_{13}\\
		a_{21}& a_{22}& a_{23}\\
		a_{31}& a_{32}& a_{33}
	\end{vmatrix}
	=
	\sum_{k=1}^3 (-1)^{1+2} a_{1k}M^1_k=
	a_{11}
	\begin{vmatrix}
		a_{11}& a_{12}\\
		a_{21}& a_{22}\\
	\end{vmatrix}
	-
	a_{12}
	\begin{vmatrix}
		a_{23}& a_{24}\\
		a_{33}& a_{33}
	\end{vmatrix}
	+
	a_{13}
	\begin{vmatrix}
		a_{21}& a_{22}\\
		a_{31}& a_{32}
	\end{vmatrix}
	=\\=
	a_{11}(a_{22}a_{33} - a_{32}a_{23}) - a_{12}(a_{21}a_{33})% Дописать
\end{gather*}\\
\newpage
Понятие минора для любого элемента определителя\\
$M_j^i$ - дополнительный минор элемента $a_{1j}$\\
% Дописать
Разложение определителя по строке $№i$
\begin{gather*}
	|A| = 
	\sum_{k=1}^n (-1)^{i+k} a_{ik}M^i_k 
\end{gather*}\\
Разложение определителя по столбцу $№j$
\begin{gather*}
	|A| = 
	\sum_{k=1}^n (-1)^{k+j} a_{ik}M^k_j 
\end{gather*}\\
\begin{gather*}
	M_j^i(-1)^{i+j} = A_j^i
\end{gather*}\\
% Название сл формулы
Сумма произведения элементов на их алгебраические дополнения:
\begin{gather*}
	|A|=\sum^n_{k+1}a_{kj}A_j^k
\end{gather*}\\
Подсчитать определитель 4-го порядка
\begin{gather*}
	\begin{vmatrix}
		21& 4& -7& 1\\
		0& 0& 1& 0\\
		3& -1& 0& 1\\
		4& 1& 0& 2
	\end{vmatrix}
	=
	(-1)^{2+3}\times1\times
	\begin{vmatrix}
		21& 4& 1\\
		3& -1& 1\\
		4& 1& 2
	\end{vmatrix}
	=
	-(-42+16+3+4-24-21)
\end{gather*}\\

\subsubsection{Ранг матрицы}
$[A] - m\times n$\\
Минор порядка $r$ для данной матрицы- это определитель, образованный  элементами данной матрицы, расположенными в некоторых выбранных $r$ строках и $r$ столбцах\\
Строки $i_1, i_2, \dots i_r$\\
Столбцы $j_1, j_2, \dots j_n$
$L$ - минор по $r$ в матрице $|A|$\\
\begin{gather*}
	L^{i_1, i_2, \dots i_r}_{j_1, j_2, \dots j_n}=
	\begin{vmatrix}
		a_{i_1g_1}& a_{i_2g_1}& a_{i_3g_1}\\
	\end{vmatrix}
\end{gather*}













































\end{document}
