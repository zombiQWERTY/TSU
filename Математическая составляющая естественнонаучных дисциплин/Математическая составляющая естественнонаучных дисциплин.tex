\documentclass[a4paper, 11pt, oneside]{article}
\usepackage[left=2cm,right=2cm,top=2cm,bottom=2cm,bindingoffset=0cm]{geometry}
\geometry{a4paper}
\usepackage{ucs,graphicx,amssymb,amsmath,mathtext}

\usepackage{tabularx}
\usepackage[utf8x]{inputenc}
\usepackage[russian]{babel}

\sloppy


\title{Математическая составляющая естественнонаучных дисциплин}
\author{Павел Зиновьев}

\begin{document}
\maketitle
\begin{abstract}
	Математик – это тот, кто умеет находить аналогии между утверждениями. Лучший 
	математик – кто устанавливает аналогии доказательств. Более сильный может заметить 
	аналогии теорий. Но есть и такие, кто между аналогиями видит аналогии.
	\begin{flushright}
		 Стефан Банах \copyright
	\end{flushright}
\end{abstract}
\newpage
\tableofcontents
\contentsname
\newpage


\section{Информация}
Преподаватель: Зотова Светлана Вячеславовна (каб. 12-314)
\subsection{Типовой рассчет}
Срок сдачи: 2 (I подгруппа) и 32 (II подгруппа)  ноября
\subsubsection{Требования}
Подписанная тетрадь 12 листов\\
ФИО, номер группы, дисциплина, преподаватель


\section{Задание на дом}


Повторить формулы работы со степенями и с корнями































































\end{document}
