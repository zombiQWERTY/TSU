\documentclass[a4paper, 11pt, oneside]{article}
\usepackage[left=2cm,right=2cm,top=2cm,bottom=2cm,bindingoffset=0cm]{geometry}
\geometry{a4paper}
\usepackage{ucs,graphicx,amssymb,amsmath,mathtext}

\usepackage{tabularx}
\usepackage[utf8x]{inputenc}
\usepackage[russian]{babel}

\sloppy


\title{История}
\author{Зиновьев Павел}

\begin{document}
\maketitle
\begin{abstract}
	Первая задача истории - воздержаться от лжи, вторая - не утаивать правды, третья - не давать никакого повода 
	заподозрить себя в пристрастии или в предвзятой враждебности.
	\begin{flushright}
		 Цицерон \copyright
	\end{flushright}
\end{abstract}
\newpage
\tableofcontents
\newpage

\section{Информация}
Вепренцева Татьяна Алексеевна (каб. 5-228)\\
Номер телефона кафедры: 25-46-00\\
Сайт с научными журналами по различным дисциплинам - elibrary.ru\\
Семинары раз в 2 недели по средам
\subsection{Книги}
Седов В. Б. - происхождение истории ранних славян\\
Греков Б. Д. - Киевская Русь\\
Рыбаков Б. А.\\
Фроянов И. А


\subsection{Темы семинарских занятий}

\begin{enumerate}
\item Этногенез восточных славян. Становление древнерусской государственности в IX-XI вв. Восточные славяне первые век нашей эры. Вопрос о прародине восточных славян
\item Предпосылки образования Киевской Руси
\item Дискуссии о проблеме происхождения власти в Киевской Руси (Нормандская теория)
\item Древнерусское государство по Русской Правде
\item Принятие христианства и его значение
\end{enumerate}

Россия - страна восточного типа


\section{История России. Введение в предмет}

История - рассказ о прошлом. Это одна из общественных наук, изучающая прошлое и настоящее развитие человечества во всем его многообразии, выявляет закономерности этого развития. Это социальная наука, опирается на исторические источники и данные вспомогательных исторических дисциплин\\
Исторические источники: вещественные, археологические раскопки, письменные (бумага, пергамент, стена в храме), лингвистические источники, фонетические источники и прочее\\\\
Вспомогательные исторические дисциплины:
\begin{itemize}
	\item Геральдика - наука о гербах
	\item Нумизматика - наука о монетах
	\item Сфрагистика - наука о печатях (документах)
	\item Метрология - наука о единицах измерения
	\item Генеалогия - наука о родах и кровно-родственных связях
	\item Историческая география - изучает как менялась карта на протяжении различных исторических периодов
\end{itemize}
В советский период применялся формационный подход в изучении истории. Его изобрел Маркс, взяли на вооружении советские школы после обработки Энгельсом\\\\
\newpage
Формации:
\begin{itemize}
	\item Первобытность
	\item Рабовладение
	\item Феодализм
	\item Капитализм
	\item Социализм
\end{itemize}

\subsection{Появление истории}
\subsubsection{Как появилась история}
Историография истории России - это описание российской истории и исторической литературы (история исторической науки в целом). Научное освещение истории начинается с IIXX в.

\subsubsection{Историки}
Начал все В. Н. Татищев (первый крупный дворянский историк). Он написал труд под названием 
<<История российская с самых древнейших времен>>. Ему принадлежит концепция общественного познания. Считал, что лучшая форма правления - монархическая.\\
М. Ю. Ломоносов (1811-1865) - автор ряда трудов (<<Краткий российский летописец>>, <<Древняя российская история>>) положил начало Нормандской теории\\
Первый капитальный труд появился при Александре I, написал Карамзин\\
С. М Соловьев - русский историк, профессор московского университета, автор энциклопедии <<История России с древнейших времен>>. Придерживался принципа историзма, старался быть как можно более объективным, не вставлял в труды политику\\
Б. Н. Чичерин (1828 - 1904) - создатель государственной школы\\
В. О. Ключевский - крупнейший историк, придавал большое значение колонизации
\subsection{Периоды и периодизация русской истории}
\begin{enumerate}
	\item Древняя Русь (IX-XI вв.)
	\item Период самостоятельных феодальных государств Древней Руси (XII-XV вв.)
	\item Русское и Московское государство (XV-XVII вв.)
	\item Российская империя периода абсолютизма (XVIII-mid IXX вв.)
	\item Российская империя периода перехода к буржуазной монархии (IXX-mid XX-begin вв.)
	\item Россия в период буржуазно-демократической республики (февраль-октябрь 1917 г.)
	\item Период становления советской государственности (1918-1920 гг.)
	\item Переходный период и период НЭПа (1921-1930 гг.)
	\item Период государственно-партийного социализма (1930-1960-mid гг.)
	\item Период кризиса социализма (1960-1990 гг.)
	\item Современный период
\end{enumerate}
\newpage

\section{Этногенез восточных славян (VI-VIII вв.). Древнерусское гос-во IX-XIII вв.}

Праславяне - земледельческо-пастушеческие племена, проживали на территории центральной и восточной Европы. Слабо изучены. Умерших кремировали, языческая особенность, характерная для них, была до времен Вятичей\\
Название славяне появилось не сразу. Римские авторы писали о венедах. Это были предки западных славян. Византия называла их славянами, антами, славинами (о восточных). В VII - VIII вв. название славяне вытеснило все остальные названия. 
\subsection{Племенные союзы}
\begin{itemize}
	\item Ильменские славене - проживали в районе озера Ильмень
	\item Кривячи - западная Двина (кровные, родные - кривячи)
	\item Полочане - часть кривячей по реке Полота, верховья западной Двины
	\item Дряговичи - (дрягва - болото)
	\item Радимечи
	\item Вятичи - верховье Оки, около Москвы-реки
	\item Балты
	\item Древляне - Припять, Днепр (древние)
	\item Поляне - район Киева, стали центром древнерусской государственности
	\item Северяне
	\item Уличи
	\item Тиверцы
	\item Белые хорваты
	\item Мужане
\end{itemize}

\subsection{Соседи восточных славян}
\begin{itemize}
	\item Дунайская Булгария
	\item Византийская империя
	\item Хазарский кагонат
	\item Мадьяры (венгры)
	\item Печенеги
	\item Волжско-камская Булгария (Болгария) - мусульмане (турки)
\end{itemize}
Славяне занимались скотоводством\\
Первое одомашенное животное --- свинья
\newpage

\subsection{Языческая религия}
Восточные славяне были язычниками, у них было дохера богов, причем с разными названиями\\
Языческая религия --- религия многобожья\\
Основу социального устройства составляла родовая община (объединение кровных родственников, совместно решавших важнейшие вопросы на общих сборах (вече))\\
Формируется соседская, или территориальная община, называлась мир (вервь)\\
Далее возникает частная (семейная) собственность вместо общин. Изменяется социум, появляется боярство (владело землями, передаваемыми по наследству)\\
Для защиты от врагов выбирали военного предводителя --- князя (конязя --- сидящий на коне). Опирался на дружину\\
У дружины была личная свобода\\
Бало народное ополчение, вечевой порядок решения важных дел\\
VI-VIII вв. --- складываются предпосылки формирования государственности\\
Восточные славяне в VI-VIII вв. представляли собой земледельческие племена, которые находились на стадии разложения родо-племенного строя и постепенно у них складывались предпосылки для образования государственности

\section{Образование гос-ва. Нормандская теория. Политическая деятельность князей IX-XI вв.}



































































\end{document}
