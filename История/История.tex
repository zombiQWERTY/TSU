\documentclass[a4paper, 11pt, oneside]{article}
\usepackage[left=2cm,right=2cm,top=2cm,bottom=2cm,bindingoffset=0cm]{geometry}
\geometry{a4paper}
\usepackage{ucs,graphicx,amssymb,amsmath,mathtext}

\usepackage{tabularx}
\usepackage[utf8x]{inputenc}
\usepackage[russian]{babel}

\sloppy


\title{История}
\author{Зиновьев Павел}

\begin{document}
\maketitle
\begin{abstract}
	Первая задача истории - воздержаться от лжи, вторая - не утаивать правды, третья - не давать никакого повода 
	заподозрить себя в пристрастии или в предвзятой враждебности.
	\begin{flushright}
		 Цицерон \copyright
	\end{flushright}
\end{abstract}
\newpage
\tableofcontents
\contentsname
\newpage

\section{Информация}
Вепренцева Татьяна Алексеевна (каб. 5-228)\\
Номер телефона кафедры: 25-46-00\\
Сайт с научными журналами по различным дисциплинам - elibrary.ru\\
Семинары раз в 2 недели по средам
\subsection{Книги}
Седов В. Б. - происхождение истории ранних славян\\
Греков Б. Д. - Киевская Русь\\
Рыбаков Б. А.\\
Фроянов И. А


\subsection{Темы семинарских занятий}

\begin{enumerate}
\item Этногенез восточных славян. Становление древнерусской государственности в IX-XI вв. Восточные славяне первые век нашей эры. Вопрос о прародине восточных славян
\item Предпосылки образования Киевской Руси
\item Дискуссии о проблеме происхождения власти в Киевской Руси (Нормандская теория)
\item Древнерусское государство по Русской Правде
\item Принятие христианства и его значение
\end{enumerate}

Россия - страна восточного типа






























\end{document}
