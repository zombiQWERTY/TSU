\documentclass[a4paper, 11pt, oneside]{article}
\usepackage[left=2cm,right=2cm,top=2cm,bottom=2cm,bindingoffset=0cm]{geometry}
\geometry{a4paper}
\usepackage{ucs,graphicx,amssymb,amsmath,mathtext}

\usepackage{tabularx}
\usepackage[utf8x]{inputenc}
\usepackage[russian]{babel}

\sloppy


\title{Основы черчения и технический рисунок}
\author{Зиновьев Павел}

\begin{document}
\maketitle
\begin{abstract}
	Когда я был юношей, где-то прочитал: «Великий Господь сказал: 
	всё сложное ненужно, всё нужное - просто». Под этим девизом и изобретаю. 
	Солдат академий не кончает, ему надо просто и надёжно
	\begin{flushright}
		 Михаил Калашников \copyright
	\end{flushright}
\end{abstract}
\newpage
\tableofcontents
\newpage

\section{Информация}
Архангельская Наталья Николаевна (каб. 9-205)\\
Чертить 1-2 раза в неделю, чтобы все успевать\\
Зачетная работа без оценки - жопу рвать не нужно\\
\subsection{Работы}
\begin{enumerate}
	\item Линии и надписи на чертежах (A3)\\
		(Гост на линии, штриховки итд)\\
		Срок: 2 недели на оценку 5
	\item Построение вписанных многоугольников. Рассматривается каким образом происходит
		деление окружности на n количество элементов
	\item Сопряжение\\
		Срок: 2 недели
	\item Ортогональный чертеж\\
		Срок: 3 недели
	\item Технический рисуок\\
		Придание объема\\
		Лист А3\\
		Срок: 2 недели
	\item Построение модели по описанию. Плоскостной и ортогональный чертежи
	\item Зачетная работа по техническому рисунку
\end{enumerate}
\subsection{Книги}
Бороткин, Бескровный, Шишмарёв - Геометрические построения\\
Чекмарёв - Справочник машиностроительному черчению\\
Вышнепольский - Справочник по черчению для ВУЗов
\subsection{Инструменты}
\begin{enumerate}
	\item Циркуль (хороший)
	\item Карандаши (2B, 2F, \{3B, 3F - заточить лопаткой\}) kohinoor (H, HB карандаши не нужны)
	\item Наждачная бумага (мягкая)
	\item Линейки: линейки, треугольник, рейсшина (инерционная линейка дома, обычная на паре)
	\item Лекало
	\item Бумага: черновая, госзнаковский ватман
	\item Ластик мягкий, приятный, белый, мало катышков
\end{enumerate}
\subsection{Аттестации}
\begin{enumerate}
	\item Конец октября. Работа по самостоятельному выбору
\end{enumerate}
\newpage





\section{Линии и надписи на чертежах}
Единая система конструкторской документации (ЕСКД)\\
Группа ГОСТов 2.300\\\\
ГОСТы. Виды ГОСТов:
\begin{itemize}
	\item ГОСТ
	\item ТУ
	\item МН
	\item СНиП
	\item ОСТ
\end{itemize}
Расшифровка стандарта:\\
Пример: ГОСТ 2301-2011
\begin{itemize}
	\item ГОСТ - Тип стандарта
	\item 2 - номер комплекса стандартов
	\item 3 - номер группы стандартов (правила выполнения технических чертежей)
	\item 01 - порядок ГОСТов в группе стандартов (в данном случае в группе 300)
	\item 2011 - год принятия ГОСТа
\end{itemize}
ГОСТ 2301-2011 - Форматы\\
Типы форматов:
\begin{itemize}
	\item A0: базовый. 1м$^2$ ($841\times1189$)
	\item A1: $594\times841$
	\item A2: $420\times594$
	\item A3: $297\times420$
	\item A4: $210\times297$
\end{itemize}

\subsection{Основная надпись}
2.104-2006 - ГОСТ на основную надпись\\
Размеры основной надписи:\\
Слева: 210\\
Справа: 25\\
Ширина: 185\\
Высота: 55\\
\subsubsection{Заполнение основной надписи}
Шрифт: 3.5\\
Правый верхний угол:\\
Вариант №1:\\
В, № - 7 шрифт\\
ариант - 5 шрифт\\
ТулГУ, гр. 132151 - 5(3.5) шрифт\\\\
ГОСТ 2.302-2011 - ГОСТ масштабов\\
ГОСТ 2.303-2011 - ГОСТ линий
\begin{enumerate}
	\item Сплошная толстая основная (линии поля чертежа, основная надпись, линии видимого контура). 0.5..1.4
	\item Сплошная тонкая. В два раза тоньше сплошной толстой линии (размерные, выносные, 
		линии штриховки, линии связи)
	\item Сплошная волнистая. Кривая Безье. Толщина произвольная, но тонкая лучше смотрится
		(линия обрыва, линия соединения части вида и разреза)
	\item Штрих-пунктирная линия. Состоит из 2х элементов: штрих и пунктир. Штрих (от 5мм до 30мм),
		расстояние между штрихами 5мм, длина пунктира 1-2мм (осевые, центровые, линии симметрии). 
		Пересекаются линии по штриху. Вырождается в прямые сплошные
		тонкие две линии, если диаметр окружности <10мм
	\item Штриховая. Расстояние штриха 2-8мм, расстояние между ними 1-2мм
	\item Штриховка под углом 45º (на ортогональных чертежах). Шаг от 1мм до 10мм. У нас 5-7мм
\end{enumerate}
\subsection{ГОСТ 2.304-81 - Шрифт чертежный}
Линейка шрифтов:
\begin{itemize}
	\item 1.8
	\item 2.5
	\item 3.5
	\item 5
	\item 7
	\item 10
	\item 14
	\item 20
	\item 28
	\item 40
\end{itemize}

\subsection{Задание на дом}
Начертить задание линии и надписи на чертежах
































\end{document}
