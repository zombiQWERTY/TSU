\documentclass[a4paper, 11pt, oneside]{article}
\usepackage[left=2cm,right=2cm,top=2cm,bottom=2cm,bindingoffset=0cm]{geometry}
\geometry{a4paper}
\usepackage{ucs,graphicx,amssymb,amsmath,mathtext}

\usepackage{tabularx}
\usepackage[utf8x]{inputenc}
\usepackage[russian]{babel}

\sloppy


\title{Математика}
\author{Павел Зиновьев}

\begin{document}
\maketitle
\begin{abstract}
	Математик – это тот, кто умеет находить аналогии между утверждениями. Лучший 
	математик – кто устанавливает аналогии доказательств. Более сильный может заметить 
	аналогии теорий. Но есть и такие, кто между аналогиями видит аналогии.
	\begin{flushright}
		 Стефан Банах \copyright
	\end{flushright}
\end{abstract}
\newpage
\tableofcontents
\contentsname
\newpage


\section{Информация}
Преподаватель: Христич Дмитрий Викторович (каб. 12-314)
\subsection{Книги}

Дмитрий Кузнецов - Сборник задач по высшей математике\\
Под редакцией Ефимова, Демидович - сборник задач для {\itshape ВТУЗ}ов\\
Данко, Попов, Кожевникова - Высшая математика в упражнениях и задачах\\
Клетеник - сборник задач по аналитической геометрии

\section{Матрицы}
Матрицей размера $m\times n$ называется прямоугольная таблица чисел, состоящая из $m$ строк и $n$ столбцов. Числа этой таблицы называются элементами матрицы. Матрицы обозначаются $A, B, C,..$ или $a_{ij}, b_{ij}, c_{ij},..$\\
Первый индекс - номер строки, второй индекс - номер столбца

\subsection{Сложение матриц}

Суммой двух матриц $A, B$ одинакового размера $m\times n$ называется матрица $C = A+B$ того же размера, каждый элемент которой равен сумме соответствующих элементов матриц $(a), (b)$
$$
	c_{ij} = a_{ij} + b_{ij}
$$
\begin{gather*}
	\begin{pmatrix}
		2& 0& -1\\
		3& 4& 5
	\end{pmatrix}
	+
	\begin{pmatrix}
		-3& 6& 8\\
		4& -2& 7
	\end{pmatrix}
	=
	\begin{pmatrix}
		2+(-3)& 0+6& -1+8\\
		3+4& 4+(-2)& 5+7
	\end{pmatrix}
	=
	\begin{pmatrix}
		-1& 6& 7\\
		7& 2& 12
	\end{pmatrix}
\end{gather*}

\subsection{Вычитание матриц}

Разностью двух матриц $A, B$ одинакового размера $m\times n$ называется матрица $C = A-B$ того же размера, каждый элемент которой равен разности соответствующих элементов матриц $(a), (b)$
$$
	c_{ij} = a_{ij} - b_{ij}
$$
\begin{gather*}
	\begin{pmatrix}
		2& 0& -1\\
		3& 4& 5
	\end{pmatrix}
	-
	\begin{pmatrix}
		-3& 6& 8\\
		4& -2& 7
	\end{pmatrix}
	=
	\begin{pmatrix}
		2-(-3)& 0-6& -1-8\\
		3-4& 4-(-2)& 5-7
	\end{pmatrix}
	=
	\begin{pmatrix}
		5& -6& -9\\
		-1& 6& -2
	\end{pmatrix}
\end{gather*}

\subsection{Умножение матриц}

При умножении матрицы $A$ на действительное число $\alpha$ получается матрица $B=\alpha A$ того же размера
$$
b_{ij}=\alpha a_{ij}
$$
\begin{gather*}
	3 \times
	\begin{pmatrix}
		-2& 5\\
		7& -6\\
		1& 3
	\end{pmatrix}
	=
	\begin{pmatrix}
		-2\times 3& 5\times 3\\
		7\times 3& -6\times 3\\
		1\times 3& 3\times 3
	\end{pmatrix}
	=
	\begin{pmatrix}
		-6& 15\\
		21& -18\\
		3& 9
	\end{pmatrix}
\end{gather*}
\begin{gather*}
	\begin{pmatrix}
		2& 5\\
		-3& 4
	\end{pmatrix}
	+ 2 \times
	\begin{pmatrix}
		-7& 2\\
		1& 0
	\end{pmatrix}
	-4 \times
	\begin{pmatrix}
		8& 15\\
		6& 20
	\end{pmatrix}
	=
	\begin{pmatrix}
		2& 5\\
		-3& 4
	\end{pmatrix}
	+
	\begin{pmatrix}
		14& 4\\
		2& 0
	\end{pmatrix}
	-
	\begin{pmatrix}
		32& 60\\
		24& 80
	\end{pmatrix}
	=
	\begin{pmatrix}
		-44& -51\\
		-25& -76
	\end{pmatrix}
\end{gather*}

\subsection{Транспонирование матриц}

Транспонирование матрицы состоит в замене строк матрицы её столбцами с соответствующими номерами

\begin{gather*}
	B = A^T\\
	b_(ij) = a_(ji)
	\begin{pmatrix}
		1&-2&4\\
		-3&5&8
	\end{pmatrix}
	^T
	=
	\begin{pmatrix}
		1&-3\\
		-2&5\\
		4&8
	\end{pmatrix}
\end{gather*}

\subsection{Произведение матриц} % Не понятно

Произведением матриц A и B называется матрица C, каждый элемент которой равен\\
$c_{ij} = \sum\limits_{l=1}^n \times a_{il} \times b_{lj}$
Для выполнения умножения матриц требуется чтобы количество столбцов первого множителя было равно количеству строк второго множителя

\begin{gather*}
	\begin{pmatrix}
		1& 2& 5\\
		4& 7& -1
	\end{pmatrix}
	\times
	\begin{pmatrix}
		2& -3\\
		6& 8\\
		-2& 1
	\end{pmatrix}
	=
	\begin{pmatrix}
		1\times 2+2\times 6+5\times (-2)& 1\times (-3)+2\times 8+5\times 1\\
		4\times 2+7\times 6+(-1)\times (-2)& 4\times (-3)+7\times 8+(-1)\times 1
	\end{pmatrix}
	=\\=
	\begin{pmatrix}
		4& 18\\
		52& 43
	\end{pmatrix}
\end{gather*}

\begin{gather*}
	\begin{pmatrix}
		2& -3\\
		6& 8\\
		-2& 1
	\end{pmatrix}
	\times
	\begin{pmatrix}
		1& 2& 5\\
		4& 7& -1
	\end{pmatrix}
	=
	\begin{pmatrix}
		2\times 1+(-3)\times 4& 2\times 2+(-3)\times 7& 2\times 5+(-3)\times (-1)\\
		6\times 1+8\times 4& 6\times 2+8\times 7& 6\times 5+8\times (-1)\\
		-2\times 1+1\times 4& -2*2+1\times 7& -2\times 5+1\times (-1)
	\end{pmatrix}
	=\\=
	\begin{pmatrix}
		-10& -17& 13\\
		38& 68& 22\\
		2& 3& -11
	\end{pmatrix}
\end{gather*}
\\
\begin{center}
	{\itshape Свойство умножения матриц:}
\end{center}
\begin{gather*}
	A\times B \not= B\times A\\
	(A\times B)\times C = A\times (B\times C)
\end{gather*}

\begin{gather*}
	\begin{pmatrix}
		0& 0& 1\\
		1& 1& 2\\
		2& 2& 3\\
		3& 3& 4
	\end{pmatrix}
	\times
	\begin{pmatrix}
		-1& -1\\
		2& 2\\
		1& 1
	\end{pmatrix}
	\times
	\begin{pmatrix}
		4\\
		1
	\end{pmatrix}
	=
	\begin{pmatrix}
		0+0+1& 0+0+1\\
		-1+2+2& -1+2+2\\
		-2+4+3& -2+4+3\\
		-3+6+4& -3+6+4
	\end{pmatrix}
	\times
	\begin{pmatrix}
		4\\
		1
	\end{pmatrix}
	=\\=
	\begin{pmatrix}
		1& 1\\
		3& 3\\
		5& 5\\
		7& 7\\
	\end{pmatrix}
	\times
	\begin{pmatrix}
		4\\
		1
	\end{pmatrix}
	=
	\begin{pmatrix}
		4+1\\
		12+3\\
		20+5\\
		28+7
	\end{pmatrix}
	=
	\begin{pmatrix}
		5\\
		15\\
		25\\
		35
	\end{pmatrix}
\end{gather*}

{\itshape Если количество строк и столбцов совпадает, то такая матрица называется квадратной}


\subsection{Определители}

Если в матрице есть чето там, то определитель и детерминант. Формула для вычисление определителей 2-го и 3-го порядков.\\
Главная диагональ идет из левого верхнего угла в правый нижний. Побочная диагональ идет из правого верхнего угла в левый нижний.

\subsubsection{Определитель квадратной матрицы 2-го порядка}
Определитель квадратной матрицы 2-го порядка:
\begin{gather*}
	|A| = \det A = 
	\begin{vmatrix}
		a_{11}& a_{12}\\
		a_{21}& a_{22}
	\end{vmatrix}
	=
	a_{11}\times a_{22}-a_{12}\times a_{21}
\end{gather*}
Из произведения элементов главной диагонали вычесть произведение элементов побочной диагонали

Примеры:
\begin{gather*}
	\begin{vmatrix}
		-1& 4\\
		-5& 2
	\end{vmatrix}
	=
	-1\times 2-4\times (-5)=18
\end{gather*}

\begin{gather*}
	\begin{vmatrix}
		a+b& a-b\\
		a-b& a+b
	\end{vmatrix}
	=
	(a+b)^2-(a-b)^2 = a^2+2\times a\times b+b^2 - (a^2-2\times a\times b+b^2)
	=\\=
	a^2+2\times a\times b+b^2-a^2+2\times a\times b-b^2=4\times a\times b
\end{gather*}
\begin{gather*}
	\begin{vmatrix}
		\cos \alpha& \sin \alpha\\
		-\sin \alpha& \cos \alpha
	\end{vmatrix}
	=
	\cos^2 \alpha - \sin \alpha \times (-\sin \alpha) = \cos^2\alpha + \sin^2\alpha = 1
\end{gather*}

\subsubsection{Определитель квадратной матрицы 3-го порядка}
Определитель квадратной матрицы 3-го порядка:
\begin{gather*}
	|A|=
	\begin{vmatrix}
		a_{11}& a_{12}& a_{13}\\
		a_{21}& a_{22}& a_{23}\\
		a_{31}& a_{32}& a_{33}
	\end{vmatrix}
	=
	a_{11}\times a_{22}\times a_{33}+a_{12}\times a_{23}\times a_{31} + a_{31} 
\end{gather*}

%Списать из тетради картинки и произведения диагоналей (про определитель)


\begin{gather*}
	\begin{vmatrix}
		3& 4& -5\\
		8& 7& -2\\
		2& -1& 8
	\end{vmatrix}
	=3\times 7\times 8+4\times (-2)\times 2+8
	\times\\\times
	(-1)\times (-5)-(-5)\times 7\times 2-4\times 8\times 8-(-2)\times (-1)\times 3=0
\end{gather*}


\begin{gather*}
	\begin{vmatrix}
		3& -2& 1\\
		1& 5& -2\\
		-1& 2& -1
	\end{vmatrix}
	=3\times5(-1)+1\times2\times1+(-2)\times(-2)\times1-1\times5\times(-1)-2
	\times\\\times
	(-2)\times3-1\times(-2)\times(-1)=-15+2-4+5+12-2=-2
\end{gather*}

\begin{gather*}
	\begin{vmatrix}
		2& 2& -1\\
		1& 1& -2\\
		5& -3& 1
	\end{vmatrix}
	=2+(-20)+3+5-12-2=-24
\end{gather*}

\begin{gather*} % Дорешать надо
	\begin{vmatrix}
		1& 2& 3\\
		4& 5& 6\\
		7& 8& 9
	\end{vmatrix}
	=0
\end{gather*}

\subsection{Метод Краймера}
\begin{gather*}
	\begin{cases}
		3x-4y=1,\\
		3x+4y=18
	\end{cases}\\
	\varDelta =
	\begin{vmatrix}
		3& -4\\
		3& 4
	\end{vmatrix}
	=
	12+12=0\\
	\varDelta_1 =
	\begin{vmatrix}
		1& -4\\
		18& 4
	\end{vmatrix}
	=76\\
	\varDelta_2 =
	\begin{vmatrix}
		3& 1\\
		3& 18
	\end{vmatrix}
	=51\\
	x = \frac{76}{24}=2\frac{1}{8}
\end{gather*}
\begin{gather*}
	\begin{cases}
		2x-y=1,\\
		4x+2y=3
	\end{cases}\\
	\varDelta = 0\\
	\varDelta_x =
	\begin{vmatrix}
		1& -1\\
		38& -2
	\end{vmatrix}
	=1\not= 0\\
\end{gather*}
Значит, решений нет

\begin{gather*}
	\begin{cases}
		3x+y=5,\\
		-6x-2y=-10
	\end{cases}\\
	\varDelta =0\\
	\varDelta_x =0\\
	\varDelta_y =0\\
\end{gather*}
Значит, бесконечно много решений (прямые лежат друг на друге)

\begin{gather*}
	\begin{cases}
		7x_1+2x_2+3x_3=15,\\
		5x_1-3x_2+2x_3=15,\\
		10x_1-11x_2+5x_3=36
	\end{cases}\\
	\varDelta =
	\begin{vmatrix}
		7& 2& 3\\
		5& -3& 2\\
		10& -11& 5
	\end{vmatrix}=-105-165+40+90+154+50=-36\\
	\varDelta_{x} =
	\begin{vmatrix}
		15& 2& 3\\
		15& -3& 2\\
		36& -11& 5
	\end{vmatrix}=-225-\dots=-72\\
	\varDelta_{y} =
	\begin{vmatrix}
		7& 15& 3\\
		5& 15& 2\\
		10& 36& 5
	\end{vmatrix}=525+540+300-450-375-504=36\\
	\varDelta_{z} =
	\begin{vmatrix}
		7& 2& 15\\
		5& -3& 15\\
		10& -4& 36
	\end{vmatrix}=ZED\\
	x_1=\frac{\varDelta}{\varDelta_x}=\frac{-36}{-72}\\
	x_2=\frac{\varDelta}{\varDelta_y}=\frac{-36}{36}=-1\\
	x_3=\frac{\varDelta}{\varDelta_z}=\frac{-36}{ZED}\\
\end{gather*}

\begin{gather*}
	\begin{cases}
		2x_1+x_2 \dots  =5,\\
		x_1 \dots +3x_3=16,\\
		\dots 5x_2-x_3=10
	\end{cases}\\
	\varDelta =
	\begin{vmatrix}
		2& 1& 0\\
		1& 0& 3\\
		0& 5& -1
	\end{vmatrix}=0+0+0-0+1-30=-29\\
	\varDelta_{x1} =
	\begin{vmatrix}
		5& 1& 0\\
		16& 0& 3\\
		10& 5& -1
	\end{vmatrix}=0+30+0-0+16-75=-29\\
	\varDelta_{x2} =
	\begin{vmatrix}
		2& 5& 0\\
		1& 16& 9\\
		0& 10& -1
	\end{vmatrix}=-32+0+0-0+5-60=-87\\
	\varDelta_{x3} =
	\begin{vmatrix}
		2& 1& 5\\
		1& 0& 16\\
		0& 5& 10
	\end{vmatrix}=0+0+25-0-10-160=-145\\
	x_1=-1\\
	x_2=3\\
	x_3=5
\end{gather*}

\begin{gather*}
	\begin{vmatrix}
		1& 1& 1& 1\\
		1& -1& 2& 2\\
		1& 1& -1& 3\\
		1& 1& 1& -1
	\end{vmatrix}
	=
	\begin{vmatrix}
		1& 1& 1& 1\\
		0& -2& 1& 1\\
		0& 0& -2& 2\\
		0& 0& 0& -2
	\end{vmatrix}=1(-2)(-2)(-2)=-8
\end{gather*}
































































\end{document}
